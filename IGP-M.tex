\documentclass[a4paper,12pt,oneside,titlepage]{article}
\usepackage[brazilian,english]{babel}
\usepackage{tikz}
\usepackage[utf8]{inputenc}
\usepackage{graphicx,color}
\usepackage{makeidx}
\usepackage{float}
\usepackage{listings}
\makeindex
%%%%%%%%%%%%%%%%%%%%%%%%%%%%%%%%%%%%%%%%%%%%%%%%%%%%%%%%%%
\usepackage{indentfirst}

% pacotes para incrementar recursos matematicos
% amsmath -> recursos avancados de matematica
% amssymb -> symbolos e fontes adicionais (ele inclui amsfonts)
% amsthm -> teorema de demonstracao
\usepackage{amsmath,amssymb} 
\usepackage{amsthm} 

%%%%%%%%%%%%%%%%%%%%%%%%%%%%%%%%%%%%%%%%%%%%%%%
% para configurar a lista enumerada
\usepackage{enumerate}

%%%%%%%%%%%%%%%%%%%%%%%%%%%%%%%%%%%%%%%%%%%%%%%
% tabela longa que quebra entre p\'aginas
\usepackage{longtable}

% linhas duplas na tabela
\usepackage{hhline}

%%%%%%%%%%%%%%%%%%%%%%%%%%%%%%%%%%%%%%%%%%%%%%%%%
% Para incluir imagens externas no LaTeX
% caixa gr\'afica
\usepackage{graphicx}

%%%%%%%%%%%%%%%%%%%%%%%%%%%%%%%%%%%%%%
% Pacotes graficos para figuras
% usando comandos de LaTeX

% Ative o pict2e, se for usar ambiente picture
%  \usepackage{pict2e}



%%%%%%%%%%%%%%%%%%%%%%%%%%%%%%%%%%%%%%%%%%%%%%%%%%%%%%%%%%%%%%%%%%
% Caso queira indentar a primeira linha do capitulo/secao, ative-o
% \usepackage{indentfirst}

%%%%%%%%%%%%%%%%%%%%
% indice remissivo
%\usepackage{makeidx}
%\makeindex % ativar (necess\'ario)

%%%%%%%%%%%%%%%%%%%%%%%%%%%%%%%%%%%%%%%%%%%%%%%%%%%%%%%%%%%%%%%%%%%%
% para acertar margens
% No caso de PCTeX 4.0 ou anterior, os arquivos 
% geometry.sty e geometry.cfg devem ficar junto com 
% o arquivo tex em edicao por n\~ao fazer parte da instala\c{c}\~ao.

\usepackage{geometry}
% Acerto de margens:
% lmargin -> left (esquerda). Interna, se frente/verso
% rmargin -> right (direita). Externa, se frente/verso
% tmargin -> top (superior).
% bmargin -> bot (inferior).
\geometry{lmargin=3.0cm,rmargin=2.0cm,tmargin=2.5cm,bmargin=2.0cm}

\usepackage{cite}

\begin{document}

\title{Estimação IGP-M: Modelo ARIMA}
\author{Vinícius Barbosa Godinho}

\maketitle




\section*{Introdução}



A partir de uma série temporal, o intuito desse estudo é o de mostrar o tratamento dos dados, a partir do software R-Studio, bem como, escolher
o melhor modelo de previsão ARIMA para a taxa mensal do IGP-M, a
partir de 2000 e, consequentemente, estimar fora da amostra os dados
um passo à frente.

Modelos de regressão, normalmente, utilizam um método de
estimação que apresenta uma relação da variável dependente com duas
ou mais variáveis, exigindo equações múltiplas. Entretanto, os modelos
ARIMA, são autoregressivos, ou seja, utilizam apenas a série temporal
para realizar a estimativa, está baseado no comportamento da própria
variável, assim não se relaciona com nenhuma outra variável a não ser
o tempo.

Assim, a análise de séries temporais leva em consideração a forte
relação das observações e dos erros ao longo do tempo, ou seja, é
caracterizada pela violação do pressuposto de não autocorrelação. As
variáveis e os resíduos não são independentes ao longo do tempo.

\subsection*{Análise Descritiva}

Nessa seção é feita uma análise descritiva acerca da série
analisada, a princípio traçamos a trajetória dos dados brutos, gráfico \ref{igpm_bruto}.
Nessa análise é possível observar o comportamento da taxa mensal do
IGP-M a partir do ano 2000, visivelmente é detectado um outlier no ano
de 2004, a partir de 2005 percebe-se certa estabilidade. Para fazer uma
análise estatística em séries temporais é necessário verificar, através de testes, se as
variáveis possuem certas características, tais como: distribuição
normal; possuem ou não componente sazonal; tendência; e raiz
unitária.




\fontsize{10pt}{\baselineskip}\selectfont{
	\begin{figure}[h!]
		\centering
		\caption{\footnotesize Taxa Mensal IGP-M (2000-2018)}
		\input{igpm_bruto}
		\label{igpm_bruto}
		\ \footnotesize Fonte: Elaboração própria a partir de dados da FGV 
	\end{figure}
}

É possível observar a partir da tabela \ref{estat_descritiva}, que a priori, uma análise
das estatísticas descritivas demonstra que a série não deve possuir
distribuição normal, pois sua mediana não coincide com a média e a
curtose é maior que três. A série apresenta uma assimetria positiva, ou
a direita, visto que a mediana é menor que a média. Os dados abaixo
apresentam a estatística descritiva dos dados em nível, analisando o
coeficiente de variação mostra-se que a série não parece ser estável, já que este é de
126,27%.





\begin{table}[h!]
	\label{estat_descritiva}
	\caption{\footnotesize Estatística Descritiva}
	\centering
	\begin{tabular}{lc}
		\hline
		
		& IGPM-M  \\
		Média & 0,616  \\
		Mediana & 0,520  \\
		Máximo & 5,19  \\
		Mínimo & -1,100  \\
		Desvio-Padrão & 0,778  \\
		Coef. Variação & 126,27  \\
		Assimetria & 1,581  \\
		Curtose & 9,546  \\ 
		
		\hline
		\multicolumn{2}{c}{\footnotesize Fonte: Elaboração própria a partir de dados FGV.}
		
	\end{tabular}
	
\end{table}

\section*{Normalidade}

A análise de assimetria realizada pela estatística descritiva é
confirmada pelo teste de Jarque-Bera de normalidade. Neste a hipótese
nula é de que a série analisada segue uma distribuição normal. Estes
valores devem ser analisados pela distribuição Qui-quadrado para 2
graus de liberdade, visto que o teste foi elaborado a partir dos
coeficientes de curtose e de assimetria. Na série é
observado um p-valor muito pequeno, portanto rejeita-se a hipótese
nula de normalidade. 

\section*{Tratamento dos Dados}

Para determinar a melhor previsão para o IGP-M em um modelo ARIMA é
detectado características da série, as principais são a volatilidade,
correlação contemporânea. O motivo de separar a tendência de nossa
variável, neste contexto, é de separar tendências de crescimento de
longo prazo e variação sazonais de fenômenos exclusivamente cíclicos,
ou aleatórios. Observa-se a seguinte equação:

\begin{center}

$ Y_t = sazonalidade_t + tendencia_t + ciclo_t + \epsilon_t $

\end{center}




No gráfico \ref{decompose} é demonstrado a decomposição clássica da
série.


\fontsize{10pt}{\baselineskip}\selectfont{
	\begin{figure}[h!]
		\centering
		\caption{\footnotesize Decomposição Clássica IGP-M}
		\input{decompose}
		\label{decompose}
		\ \footnotesize Fonte: Elaboração própria a partir de dados da FGV 
	\end{figure}
}




\section*{Sazonalidade}

O componente sazonal, ou seja, a presença de um ciclo durante
certo período pode causar instabilidade para a série, Enders (2014),
destaca que utilizar uma série temporal ignorando o componente
sazonal pode gerar uma variância alta na regressão. A priori, uma
análise gráfica não é possível verificar a presença de sazonalidade na
série, porém vamos utilizar outras ferramentas para constatar a
ausência de sazonalidade. Segundo Gujarati (2003), um método de
verificação do componente sazonal determinístico é inserir uma matriz
de variáveis dummies mensais auxiliar de cada período de tempo
regredido no modelo. Como a série analisada é mensal, o modelo
testado pode ser descrito pela seguinte regressão:
\begin{center}

$ Y_t = \alpha + \beta_2M_2  + \beta_3M_3  + ... * \beta_11M_11  + \beta_12M_12  + \epsilon_t $

\end{center}


Nota-se que é incluído dummies para cada mês exceto o primeiro,
para não ocorrer o problema de linearidade perfeita. Feita a regressão é
efetuado um teste F, onde, $ H_o : \beta_2 = \beta_3 = ...= \beta_11 = \beta_12 = 0$ , ou seja é
testado se todos os coeficientes mensais são, se forma conjunta,
estatisticamente iguais a zero. O teste através da regressão de dummies
testa a presença de uma sazonalidade mensal determinística nos dados.

É preciso verificar os p-valores de cada mês, se estes rejeitam a
hipótese nula, é verificado que possuem componente sazonal. Todas as dummies aceitam a hipótese
nula, portanto não é necessário dessazonalisar a série. Como forma de
exercício apresentamos o gráfico \ref{sazonalidade} da série em dados brutos
versus a dessazonalizada, nota-se que a trajetória é praticamente a
mesma.

\fontsize{10pt}{\baselineskip}\selectfont{
	\begin{figure}[h!]
		\centering
		\caption{\footnotesize Comparação com IGP-M dessazonalizado}
		\input{sazonalidade}
		\label{sazonalidade}
		\ \footnotesize Fonte: Elaboração própria a partir de dados da FGV 
	\end{figure}
}

\section*{Tendência}

Existem várias maneiras de remover tendência de uma série, uma
bastante usual é através da primeira diferença, mas como a série
utilizada está em taxa, ou seja, já está diferenciada não vai ser utilizado
esse método (apenas se a série for não estacionária, que será realizado
testes mais para frente).
Se a série apresentar uma tendência determinística de longo
prazo, uma forma simples de tratamento é regredir a série contra o
tempo e recolhendo os resíduos.
Segundo Enders (2014) outro método de decompor a tendência é
através do filtro de Hodrick and Prescott (1997). Este não assume uma
tendência perfeitamente linear, e pode ser utilizado para remover
tendências não determinísticas.
Como podemos observar no gráfico \ref{tendencia} a série praticamente
não possui tendência, portanto será utilizada a série bruta.


\fontsize{10pt}{\baselineskip}\selectfont{
	\begin{figure}[h!]
		\centering
		\caption{\footnotesize Métodos de remover tendência}
		\input{tendencia}
		\label{tendencia}
		\ \footnotesize Fonte: Elaboração própria a partir de dados da FGV 
	\end{figure}
}

\section*{Outliers}

Analisando o gráfico da série, é possível verificar a presença de
fortes outliers, para cima e para baixo, no ano de 2004. Portanto a série
escolhida para a previsão é a partir de 2005, como apresentado no
gráfico \ref{igpm}.


\fontsize{10pt}{\baselineskip}\selectfont{
	\begin{figure}[h!]
		\centering
		\caption{\footnotesize Série utilizada para a previsão (2005-2018)}
		\input{igpm}
		\label{igpm}
		\ \footnotesize Fonte: Elaboração própria a partir de dados da FGV 
	\end{figure}
}

\section*{Estacionaridade}

Para o caso de séries não estacionárias, é preciso determinar a ordem de integração, d , das mesmas, para se estimar um modelo ARIMA. Em suma uma série é estacionária quando a média, variância e covariância não varia no tempo, quando esta é estacionária seu valor
converge para a média ao longo do tempo. Com uma série não
estacionária não se pode fazer previsões, pois os valores não conhecidos
podem ter caráter explosivo. Portanto, uma série estacionaria não apresenta mudança sistemática na média e na variância. As condições
para que isso ocorra são:


\begin{enumerate}[i]
	\item $ E(y_t) = \mu para todo t; $
	\item $Var(y_t) = E[(y_t-\mu)^2] para todo t; $
	\item $ Cov(y_t,y_{t-s}=Cov(y_t,y_{t+s}),s=1,2,3,...,n, para todo t $
	
\end{enumerate}


Existem vários testes de estacionaridade, a seguir utilizaremos três
dos mais usuais, o \textit Dickey-Fuller, \textit Dickey-Fuller Aumentado e \textit Phillips-Perron, apresentando a metodologia \textit Box-Jekins.   

\subsection*{O teste de \textit Dickey-Fuller}


Suponha o seguinte processo para uma série:
\begin{center}
	
$ Y_t = \varTheta Y_{t-1} + \varepsilon_t  $	
	
\end{center}

Sendo, $ \varepsilon_t \sim N(0) $, iid.


Se $ |\varTheta| < 1 $, então o processo é estacionário, $\varTheta  = 1 $ o processo
apresenta raíz unitária, assim não estacionário. O diferencial do teste
\textit Dickey-Fuller, é que a determinação da ordem de integração é feita
testando, na prática como as equações a seguir.

\begin{center}

$ Y_t - Y_{t-1} = \varTheta Y_{t-1} - Y_{t-1} + \varepsilon_t $

$ \varDelta Y_t  = (\varTheta - 1) Y_{t-1}  + \varepsilon_t $

$ \varDelta Y_t  = \pi Y_{t-1}  + \varepsilon_t $

\end{center}


A hipótese nula do teste é que $ \pi = 0 $  . Desse modo, se  $ \pi = 0 $  , então $
\varTheta  = 1 $  ,  consequentemente  possui raíz unitária. Portanto, para testar esta hipótese é rodado a regressão da primeira diferença da série contra
sua defasagem.



\begin{lstlisting}[language=R,caption={DF}]



Call:
lm(formula = diff(ig) ~ lag(ig, -1)[-length(ig)] - 1)

Residuals:
Min      1Q  Median      3Q     Max 
-1.5989 -0.2178  0.0731  0.4607  1.3939 

Coefficients:
Estimate Std. Error t value Pr(>|t|)    
lag(ig, -1)[-length(ig)] -0.27334    0.05421  -5.042 1.23e-06 ***
---
Signif. codes:  0 *** 0.001 ** 0.01 * 0.05 . 0.1   1

Residual standard error: 0.519 on 161 degrees of freedom
Multiple R-squared:  0.1364,	Adjusted R-squared:  0.131 
F-statistic: 25.42 on 1 and 161 DF,  p-value: 1.228e-06


\end{lstlisting}



Analisando o p-valor rejeita-se a hipótese nula, portanto a série é
estacionária em nível, o que corrobora com a análise gráfica e que a
série é uma taxa, portanto é como se estivesse em primeira diferença.


\subsection*{O teste de \textit{Dickey-Fuller Aumentado (ADF)}}

O teste ADF assume que os erros são não correlacionados,
portanto no caso de processos autoregressivos esta correlação terá
como consequência estimadores ineficientes e viesados. Como afirma
González-Rivera (2013), a inferência só pode ser realizada se os resíduos
não apresentarem autocorrelação, assim após realizar o teste ADF é
necessário verificar a correlação dos resíduos através dos
correlogramas. A desvantagem desse teste é determinar o número de
defasagens utilizado, pois queremos utilizar o menor número possível
de defasagens que garanta a não autocorrelação dos resíduos, nesse
teste o número de defasagens foi escolhido pelo critério AIC, abaixo
realizamos o teste com e sem constante.
 	
 	
 
 \begin{lstlisting}[language=R,caption={ADF}]
 
 
 ############################################### 
 # Augmented Dickey-Fuller Test Unit Root Test # 
 ############################################### 
 
 Test regression none 
 
 
 Call:
 lm(formula = z.diff ~ z.lag.1 - 1 + z.diff.lag)
 
 Residuals:
 Min       1Q   Median       3Q      Max 
 -1.61471 -0.20935  0.07579  0.46754  1.38930 
 
 Coefficients:
 Estimate Std. Error t value Pr(>|t|)    
 z.lag.1    -0.26714    0.05924  -4.510 1.26e-05 ***
 z.diff.lag -0.02217    0.08148  -0.272    0.786    
 ---
 Signif. codes:  0 *** 0.001 ** 0.01 * 0.05 . 0.1   1
 
 Residual standard error: 0.5221 on 159 degrees of freedom
 Multiple R-squared:  0.1366,	Adjusted R-squared:  0.1258 
 F-statistic: 12.58 on 2 and 159 DF,  p-value: 8.455e-06
 
 
 Value of test-statistic is: -4.5095 
 
 Critical values for test statistics: 
 1pct  5pct 10pct
 tau1 -2.58 -1.95 -1.62
 
 \end{lstlisting}
 
 
Portanto, em conformidade com o teste de \textit{Dickey Fuller}, rejeita-se
a hipótese nula, ou seja, a série é estacionária em nível. Como
observado no correlograma abaixo, os resíduos não possuem
autocorrelação, assim pode-se usar a inferência do teste.



\subsection*{Identificação dos modelos ARIMA}

Para a escolha do modelo, é definido o mais parcimonioso
possível, através da metodologia Box-Jenkins, o intuito é determinar os
valores (p,d,q) do modelo ARIMA, segundo Moretin (1985), o
procedimento de identificação consiste em duas partes:
\begin{itemize}
	\item diferenciar a série Yt, tantas vezes quanto necessário, para se
obter uma série estacionária, de modo que o processo seja
reduzido a um ARMA(p,q). O número de diferenças necessárias
(d) para que o processo se torne estacionário, é alcançado
quando a FAC amostral decresce rapidamente para zero;
\item identificar o processo ARMA(p,q) resultante, através da análise
das autocorrelações e autocorrelações parciais. Pode-se verificar
os gráficos das séries e das diferenças, suas médias e
variâncias, suas autocorrelações e seus correlogramas com os
respectivos intervalos de confiança. Os gráficos da FAC e FACP
estão apresentados abaixo.


\end{itemize}


Os modelos selecionados são: 

\textbf{Modelo 1:} AR(1) $ Y_t= c + \phi_1 y_{t-1} + \varepsilon_t $

\textbf{Modelo 2:} MA(4) $ Y_t= \mu + \varTheta_1 \varepsilon_{t-1} + \varTheta_2 \varepsilon_{t-2} +\varTheta_3 \varepsilon_{t-3} +\varTheta_4 \varepsilon_{t-4} +\varepsilon_t$

\textbf{Modelo 3:} ARMA(1,1)  $ Y_t= c + \phi_1 y_{t-1} + \varTheta_1 \varepsilon_{t-1} + \varepsilon_t$


\textbf{Modelo 4:} ARMA(1,4) $ Y_t= c + \phi_1 y_{t-1} + \varTheta_1 \varepsilon_{t-1} + \varTheta_2 \varepsilon_{t-2} +\varTheta_3 \varepsilon_{t-3} +\varTheta_4 \varepsilon_{t-4} +\varepsilon_t$

Para verificação dos modelos é necessário verificar se os resíduos da regressão são não autocorrelacionados. É usado o teste \textit{Ljung – Box}, este tem como hipótese nula a independência de uma dada série de tempo, ou seja, os resíduos da regressão são conjuntamente não correlacionados ao longo do tempo.

\begin{lstlisting}[language=R,caption={ADF}]

> reg1.i <- arima(ig, order = c(1,0,0))
> Box.test(resid(reg1.i), lag=6, type='Ljung-Box', fitdf=1)

Box-Ljung test

data:  resid(reg1.i)
X-squared = 6.936, df = 5, p-value = 0.2254

> reg2.i <- arima(ig, order = c(0,0,4))
> Box.test(resid(reg2.i), lag=6, type='Ljung-Box', fitdf=1)

Box-Ljung test

data:  resid(reg2.i)
X-squared = 0.64082, df = 5, p-value = 0.9861

> reg3.i <- arima(ig, order = c(1,0,1))
> Box.test(resid(reg3.i), lag=6, type='Ljung-Box', fitdf=1)

Box-Ljung test

data:  resid(reg3.i)
X-squared = 4.2473, df = 5, p-value = 0.5144

> reg4.i <- arima(ig, order = c(1,0,4))
> Box.test(resid(reg4.i), lag=6, type='Ljung-Box', fitdf=1)

Box-Ljung test

data:  resid(reg4.i)
X-squared = 0.35395, df = 5, p-value = 0.9965
\end{lstlisting}
 
Analisando o teste \textit{Box-Ljung}, em todos os modelos se aceita a hipótese nula, ou seja, os resíduos não são autocorrelacionados, portanto não descartamos nenhum modelo.

Para testar se os resíduos são normalmente distribuídos é utilizado o teste de \textit{Shapiro-Wilk}, neste a hipótese nula é que as séries analisadas são normalmente distribuídas, abaixo o resultado do teste, onde aceita a normalidade em todos os modelos. 

\begin{lstlisting}[language=R,caption={ADF}]
> shapiro.test(resid(reg1.i))

Shapiro-Wilk normality test

data:  resid(reg1.i)
W = 0.99166, p-value = 0.4623

> shapiro.test(resid(reg2.i))

Shapiro-Wilk normality test

data:  resid(reg2.i)
W = 0.99072, p-value = 0.3679

> shapiro.test(resid(reg3.i))

Shapiro-Wilk normality test

data:  resid(reg3.i)
W = 0.99081, p-value = 0.3766

> shapiro.test(resid(reg4.i))

Shapiro-Wilk normality test

data:  resid(reg4.i)
W = 0.99118, p-value = 0.4126

\end{lstlisting}

Como não descartamos nenhum modelo, é usado dois critérios de comparação, o \textit Akaike (AIC) e o Bayesiano de \textit Schwarz(BIC), quanto menor o critério maior é o poder de informação do modelo. Como é observado os critérios no AIC são bem próximos, com o melhor modelo o ARIMA(1,0,1), já pelo BIC o melhor modelo foi ARIMA(1,0,0).

\begin{lstlisting}[language=R,caption={ADF}]

> AIC(reg1.i)
[1] 236.8574
> AIC(reg2.i)
[1] 237.7422
> AIC(reg3.i)
[1] 236.4344
> AIC(reg4.i)
[1] 238.7
> BIC(reg1.i)
[1] 246.1387
> BIC(reg2.i)
[1] 256.3047
> BIC(reg3.i)
[1] 248.8094
> BIC(reg4.i)
[1] 260.3562

\end{lstlisting}

De acordo com González-Rivera(2013), uma forma de escolher o melhor modelo de previsão é através da soma do erro ao quadrado médio (SQE). Avaliando os erros de previsão para um horizonte de 11 meses, ou seja, estimação da série de janeiro de 2005 até dezembro de 2015, portanto será previsto os meses de janeiro a novembro de 2016.

\begin{lstlisting}[language=R,caption={ADF}]

> sqe1
[1] 0.2544828
> sqe2
[1] 0.2630116
> sqe3
[1] 0.2402772
> sqe4
[1] 0.2631178
> 

\end{lstlisting}

Como podemos observar o menor SQE foi do modelo 3, ARIMA(1,0,1), utilizaremos ele como o modelo de previsão um passo a frente.


\end{document}
